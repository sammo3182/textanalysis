\documentclass[abstracton,UTF8]{ctexart}
%\documentclass[12pt,abstracton]{article}
\usepackage[english]{babel}
\usepackage[letterpaper,margin=1in,footskip=.5in]{geometry} %left=0.6in,right=0.6in,top=1in,bottom=1in 用于细致调整
\usepackage[tbtags]{amsmath}  
%Place equation numbers on the left by adding [leqno]; [tbtags] ‘Top-or-bottom tags’ For a split equation, place equation numbers level with the last (resp. first) line, if numbers are on the right (resp. left).
\usepackage[scaled=.90]{helvet}
\usepackage{courier}
\usepackage{times}
\usepackage{setspace}
\usepackage{paralist}
\usepackage{titling}
\usepackage{graphicx}  %Graph
\usepackage{amssymb}  % Math
\usepackage{dsfont}
\usepackage{fancyhdr} %Header set
\usepackage{indentfirst} %indent at the first paragraph
\usepackage[sort&compress]{natbib} %3 lines from here Bibliographic; sort & compress orders multiple citations into the sequence in which they appear in the list of references+;but in addition multiple numerical citations are compressed if possible (as 3-6, 15)
\usepackage[utf8]{inputenc}
\bibpunct{(}{)}{;}{a}{}{,}
\usepackage{booktabs} %thinner lines in table
\usepackage{dcolumn}  %aligning decimal points
\usepackage[table]{xcolor}  %highlight parts of a table
\usepackage[T1]{fontenc} %dealing with "<>" unable to appear problems
\usepackage{multirow} %merge rows in table
\usepackage[labelfont=bf]{caption}  % Border the caption includiong"Figure." and "Table."
\usepackage{import} % to import file in subfolders, using \import{full path}{file}
\usepackage{rotating} %rotate tables which are too wide
\usepackage{tablefootnote} %Maintain footnotes appearing in table
\usepackage[capposition=top, font=scriptsize]{floatrow} % adding notes for the figures
%\usepackage{subfigure} %纵排二图
\usepackage{subcaption}
	\captionsetup{textfont = sl} % use slanted font shape automatically for all captions
\usepackage{appendix} %精细化附录
\usepackage[colorlinks,
            linkcolor=red,
            anchorcolor=blue,
            citecolor=black
            ]{hyperref}    %bookmark
\usepackage[capitalise,noabbrev]{cleveref} %automatically detect type of reference by label, using \cref{key} instead of \ref{key}; at the begining of sentences, use \Cref{key}
\usepackage{enumitem, hyperref}
\makeatletter
\def\namedlabel#1#2{\begingroup
	#2%
	\def\@currentlabel{#2}%
	\phantomsection\label{#1}\endgroup
}
\interfootnotelinepenalty=10000 %Keep long footnote together
\author{Yue Hu}

\def\Tiny{\fontsize{5pt}{5pt} \selectfont} %choose font smaller than \tiny, the smallest size is 5pt.

%Subtitle Codes
\makeatletter
\def\s@btitle{\relax}
\def\subtitle#1{\gdef\s@btitle{#1}}
\def\@maketitle{%
  \newpage
  \null
  \vskip 2em%
  \begin{center}%
  \let \footnote \thanks
    {\LARGE \@title \par}%
  \if\s@btitle\relax
  \else\typeout{[subtitle]}%
   \vskip .5pc
   \begin{large}%
    \textsl{\s@btitle}%
    \par
   \end{large}%
  \fi
    \vskip 1.5em%
    {\large
      \lineskip .5em%
      \begin{tabular}[t]{c}%
        \@author
      \end{tabular}\par}%
    \vskip 1em%
    {\large \@date}%
  \end{center}%
  \par
  \vskip 1.5em}
\makeatother

% This code puts the section numbering in the margins
%\renewcommand*{\othersectionlevelsformat}[1]{%
%	\makebox[0pt][r]{%
%	\csname the#1\endcsname\enskip}%
%	}

\begin{document}
	\title{The Dynamic of the Concept of Democracy in Modern China}
	\date{\today}
	\maketitle
	
	\setlength{\parindent}{2em} %段前空格

This paper purposes to show the dynamic in the construction of the concept ``democracy.'' in a society without much democratic experiment. As \citet{Diamond2006} points out, the success of democratization entails a critical condition that the social members can be accurately aware of the liberal discourse of democracy and, furthermore, criticize the non-democratic practices. This condition can be more important even after the success of democratization to maintain this achievement \citep{Almond1989,Putnam1994}. But, how does this possible in societies with little democratic experience, where, in many cases, the elites\footnote{This study defines elites mainly as the decision makers of policies of a country.}, the beneficiaries of the nondemocratic regime, dominate the construction of the political knowledge of the masses? One may argue that the nowadays global share of information can reduce such elite dominance and increase the possibility of the public to receive accurate knowledge about liberal democracy. But, there is little evidence showing that the influences of these factors are always powerful enough to overwhelm the elites' dominance on the public opinions. Then, is there also a possibility that the elite-driven conceptual construction may deliver the liberal discourse of ``democracy'' to the public?  

A positive answer to this question needs at least two conditions. First, elites associates the concept of democracy with what they are good at or what they intend to work on. Second, there is a moment elites would work on building a democratic regime, even if they are benefited from the nondemocratic one. When both happens one can expect the elites to dominate the masses' understanding of democratic towards the liberal direction. In this study, I will specially focus on the first condition in this study. Particularly, I argue that, in a society with little democratic experience, such as an authoritarian country, elites incorporate their immediate focus and advantages with the concept of democracy, rather than just using that to justify the established authoritarian regime. In terms of this, the connotation of the concept of democracy is not static but in a dynamic corresponding to the contemporary primary political missions in an authoritarian country. The empirical evidence of this argument is collected from the Chinese case. On the one hand, the democratic transformation issue of China is always a hot spot in democratization studies\citep[for instance, see ][]{Dickson2000,Dickson1997,Pei1995,Chen2002a,White1994,Howell1998,Tien1999,Zheng1999,Zheng2004}. On the other hand, in terms of the external validity of the theory, if I can find the dynamic in the conceptual construction of ``democracy'' in China, the most powerful and successful authoritarian country in the past two decades where the liberal democracy is treated as the last alternative in regime reforms, there will be a high chance that similar patterns existing in other non-democratic countries.

I conduct an empirical test to detect this dynamic based on the data from a official newspaper in modern China. I collect over a million articles published in \textit{People's Daily} the official newspapers of Chinese Communist Party (CCP, the incumbent party since 1949) in the past about 57 years (1946-2003). Then I conduct a series of Structure Topic Models (STM) on these data to seek for the relationships between the topic about democracy and other topics. The results are mixed but interesting. On the one hand, the topics suggesting the authoritarian characteristics do correlate with the topic about democracy in all four periods of data, but in different degrees. On the other hand, the authoritarian characteristic topics are rarely those correlating the democracy the most. Instead, the frequencies and contexts the official newspaper talked about democracy were very different across the time. Moreover, the most correlated topic to these ``democracy'' topics were consistently about the contemporary primary missions. In other words, the concept of ``democracy'' in authoritarian countries like China is not simply for justifying the authoritarian institutional set, such as favoring stability and people-oriented governance. The construction is involved in a dynamic to respond to the immediate foci of the ruling government. In this sense, once the elites believe they are good at it or they intend to practice it by some way, one may expect that this dynamic will deliver the accurate concept of liberal democracy to the masses. A democracy created through this way may be more solid than by other ways. This result is largely confirmed by a robustness check on a separate corpus of the \textit{Selections} of the cores of the three generations of leadership of CCP for robustness check. To my knowledge, this is the first time ever these two important sources of Chinese political propaganda and education are systematically analyzed through a scientific empirical test over that long time period. 

In the following sections, I review the existing literature of the static understanding of the democratic concept in non-democratic countries, especially China. Then, I develop an alternative theory to elaborate the dynamic of the conceptual construction of ``democracy.'' A large-corpus text analysis is conducted to test the hypotheses drawn from this theory. Finally, I discuss how this study can be helpful to understand and anticipate the political transitions in modern China.
	
\section{Understand the Concept of Democracy}\label{s:literature}

It is no doubt that there have been sufficient debates about what democracy exactly is \citep[see, for example, ][]{Sen1999,Dahl1989,Schumpeter1947,Pateman1970,Karl1991a,Munck2014,Przeworski2000}. Nevertheless, from a very broad perspective, they still share a general direction which may be called as ``liberal democracy.'' It involves a series of core values, such as competitive election, freedom of speech, political equivalence and participation, etc. The difference among these definitions of democracy largely lies on aspects they emphasized \citep{Lu2014a}. Very recently, Shi Tianjian and his colleagues made an important contribution to this literature by pointing out that the concept of democracy can be different according to the social context \citep{Lu2014a,Shi2014}. Combining survey data and common democracy quality measures, they show that the the perspective of democracy in different countries can be very diverse \citep[Table 1]{Lu2014a}. They show with more survey data that people who live in China, an authoritarian country, have essentially different understanding of democracy from the common liberal concept. Comparing to liberal democracy which is mainly based on the institutional setting of fair, competitive elections and free and equal political participations, democracy in Chinese people's mind has more ``guardianship'' characteristics. It is not about if the leaders are elected through legitimate procedures, but about if the chosen leaders are good enough to increase people's welfare. In a large sense, the quintessence of this guardianship democracy is embodied in the ``minben (people-as-the-basis) doctrine of Confucianism'' \citep[198]{Shi2014}. It ``prescribes the rulers to take care of people’s interests and pursue the collective benefit, and necessitates that rulers listen to people's opinions...Clearly, the minben doctrine identifies paternalistic meritocracy as the ideal form of government'' \citep[7]{Lu2014a}.

This argument may be tracked back to the culturalist idea founded by \citet{Weber1958} and \citet{Almond1989}, and latter developed by a series of scholarships \citep[see, for example, ][]{1984,Putnam1994,PYE2006,Pye1992,Inglehart2005,Fuchs2007}. They argue that culture does matter for democratization for creating different social environment of democratic transition and maintenance. The distinguishing contribution of Shi et al. is highlighting that the consequence of culture is not just the degree of difficulty for the social members to approve liberal democracy values and regimes, but what discourse of democracy they would approve. However, their theory also encounters the potential problems the traditional culturalist theories always have to face that based on the same cultural background, some countries and regions accept the liberal democracy. (The corresponding cases for China can be Japan, Korea, and especially Taiwan.)

The other aspect of these cases is that the elites usually dominate the public opinions there. There is also a large literature in comparative politics discussing about this framing capacity of authoritarian leaders on the public opinions and attitudes \citep{Miller1979,Jiang2015,Zheng2004,Stein2012,Shirk2010}. The underlying logic is basically that, if saying formal education and media are two major sources to construct people's political knowledge and opinions, they are both under certain degrees of control by the political elites in nondemocratic countries. By these controlling, supervising, and censoring means, the elites enable to frame the public opinions or attitudes on various political topics and issues. Understanding democracy should not be an exception. Actually, not only the elites of non-democratic countries have this mind-shaping capacity. The theory is more broadly applied to differenty types of regimes, even where the controls of formal education and media are not as powerful as in the nondemocratic countries.\citep{Patterson1999,Chan1997,Sniderman2004,Chong2007,Zaller1994,Zaller1990}. 

The cases where the liberal democracy concept can be popularly accepted in Confucian cultural based societies and the framing capacity of political elites suggest that there could be a chance in the conceptual constructions in authoritarian countries. The dynamic will allow elites to frame the public understanding with the concept of liberal democracy. To make this possible requires at least two conditions. First, elites associates the concept of democracy with what they are good at or what they intend to work on. Second, there is a moment elites would work on building a democratic regime, even if they are benefited from the nondemocratic one. When both happen, one can expect the elites to dominate the masses towards liberal, rather than guardianship, democracy, given that they know what the liberal democracy is.\footnote{This is a given assumption that I would not talk more latter. Since the elites have more chances to connect to the ``outside'' world, they can collect information about what the liberal democracy is from foreign texts and observations.} There are already some existing studies of the democratic transform relating to the second condition. Some argue that elites will approve the liberal democracy as a solution for the immediate crisis \citep{Przeworski2000,ODonnell2013,Salame1994}. Another theory could be that elites believe they are strong enough to win the election in a democratic regime \citep{Magaloni2006}. No matter which exact theory is true, they show that the situation is possible when elites are willing to deliver the liberal democracy to the masses. Borrowing the words of ``willingness-opportunity'' theory \citep{Cioffi-Revilla1995,Starr1978}\footnote{This theory is basically a descriptive version of Expected Utility mechanism.}, we have the willingness part. 

What the existing literature has not fully discussed is the opportunity part. If what Shi and his colleagues show is the dominant feature of the conceptual construction of democracy in nondemocratic countries, such opportunity would never exist. In their theory, the discourse of the concept is the winners' decision after a ``fierce contests over the institutional settings in practice.'' In this case, the discourse would not change unless there is another institutional contests and a different winner. In the following section, I offers an alternative theory about the possibility of the opportunity.



\section{An Elite-Driven Theory of the Discourse of Democracy}\label{s:theory}
My theory argues that to presenting the winning institutional setting is not the only or the most important function of constructing the discourse of democracy. Elites construct the concept also and maybe more for legitimizing their immediate primary missions---the key projects of the authority. A consequence of this is a dynamic in the discourse of democracy when the missions change. On the other hand, my theory does not deny what Shi and his colleagues found from the survey data. Instead, I incorporate their argument into a more general framework. It can be presented in a very simple formal model:
\begin{align}
D_t =& aW_t\cdot O_t, \label{e:discourse}\\
O_t =& bI + cM_t + dC_t, \label{e:opportunity}
\end{align} where $a > 0$ and $t \in T$ denotes the time periods the authority holds the power. In the \cref{e:discourse}, $ D $ is the discourse of democracy in a nondemocratic society at the certain period $t$; $ W_t$ represents the willingness of the political elites on a certain discourse of democracy at the same time; and $ O_t $ represents the opportunity to deliver that discourse to the public. In \cref{e:opportunity}, $ I $ is the consideration to presenting the winning institutional setting; $ M_t $ is primary missions the authority focuses on at time $t$; and $C_t$ capture all the other variables that may affect $ O_t $. Finally, $a,\ b,\ c,$ and $d$ are the contributions of each element in the right-hand side to the left-hand side. From \cref{e:discourse} and \cref{e:opportunity}, we can easily deduct 
\begin{equation}
D_t = aW_t\cdot (bI + cM_t + dC_t). \label{e:full}
\end{equation}

\cref{e:full} shows that the discourse of the concept of democracy in a given nondemocratic country at time $t$ involves two aspects of conditions. The second term of the right-hand side (or the $O_t$ in \cref{e:discourse}) is what topics the elites can put into the discourse of democracy. The first term represents how much they would like to present these package of discourse. For each particular topic, elites' willingness to present them are different, viz., $abW_t \neq acW_t \neq adW_t$. However the product relationshps in both \cref{e:discourse} and \cref{e:full} implies that both the willingness and the opportunity are necessary conditions for the discourse of democracy at the given time period. What Shi and his colleagues show is that $aW_t\cdot bI > 0,\ \forall t.$ It shows the entire picture only when $c = 0$ and $d = 0$.

I argue an alternative possible case where $c > b > 0$ and $c > d$. In descriptive words, holding willingness part ($W_t$) constant at $t$, the contemporary primary mission correlates the discourse of democracy the most at this period, while the consideration of showing the winning institutional setting also contribute to the discourse of democracy with other topics, but not as strong as the primary mission. There is a basic rational logic back on this. There are always multiple topics the elites would like to frame in the masses' opinions. However, assuming the power of the elites' framing capacity is constrained, they are only able to distribute this capacity unevenly to different topics. In this case, the best choice is focusing more on the most important missions they are implementing or are going to do so, rather than maintaining on the same topic, and the contemporary primary missions are usually the topics asking for legitimacy and attentions. Moreover, repeatedly shaping the same topic is bound to reduce its marginal effect, giving the majority social members are already been shaped. Therefore, a rational choice for elites is to focus more on the contemporary issues rather than always presenting the institutional setting. A consequence of this will be a dynamic in the discourse of democracy along with the shifts of the primary missions of the authority across time periods. At the same time, justifying the institutional setting is not unimportant or an once-for-good job. It also needs continuously reinforcement, especially considering the population changes in generation and other external factors. Hence, it should also be taken into account when the elites construct the discourse of democracy, although not as important as the primary missions.

To make above theory work also needs three assumptions. First, the political elites are rational. I apply this classical assumption to study the decision-making process. Although leaders may not appear using a complete rational process on every decisions they make. But, from their final choice, one can argue that they do act as if they are rational \citep{BuenodeM2003}. Second, the political elites believe the democratic concept matters to his authority. This is largely an contextual statement for modern countries after the World War II. \citet{Schedler2013} argues that the single-party regime already lost the legitimacy after the Cold War. Scholars, like \citet{Fukuyama2006} and \citet{Huntington1993}, provide theoretical and empirical evidence that the trend of democratization already started in the $19^{th}$ century. In this context, as the ruling power of a modern country, the rulers have to somewhat justify their ruling ways follows this trend of improving the human development. The third assumption is that political elites has the ability to shape the opinions of their agents and the public. As discussed in \cref{s:literature}, there have been theoretical and empirical evidence to support this assumption.

If all these assumptions are held, my elite-driven theory about the discourse of democracy can infer the following hypotheses:
\begin{itemize}
	\item[\namedlabel{h:dynamic}{$H_1$}]: The discourse of democracy varies across the periods if the primary missions of governance is in change. 
	\item[\namedlabel{h:response}{$H_2$}]: The democracy topic correlate the topic among the primary mission the most in the political discourse of the authority.
\end{itemize}


I design an empirical test to examine the above hypotheses. However, because of the unique characteristics of the topics, it would be difficult (and also risky) to conduct a regular statistical test. Instead, I develop a test based on the STM method. In the following section, I will discuss this design and also the data selection issue.

\section{Design and Data}
Since the unit of the theory is topics and discourse, I am lying in the area of content analysis. A difficulty to apply regular statistical methodology in content analysis relates to the data features. In content analysis, the data consist not of multiple data points but articles. The data set is usually called ``corpus'' \citep{Jockers2013} in the linguistic or digital humanity term. In this case, it is almost impossible to run any regressive analyses on them unless the characteristics of each articles can be generalized to categories or orders. Generally, there are two types of approach that can solve this problem to transfer the text format data to data that are eligible for statistical analysis, that is, human coding and computer-assisted automated analysis \citep{Lucas2015,Blei2012,Roberts2014}. The latter method is favored by this study because of the following three reasons.

Most importantly, the automated method based on probability topic models allows me to discover topics in the corpus\footnote{A term in linguistics and digital humanity for text datasets \citep{Jockers2013}} rather than pre-classify them, and do hypothesis tests about the correlations between topics \citep{Roberts2013}. This method can highly reduce the measurement error caused by the inconsistent subjective judgments. Moreover, the approach allows me to analyze the influence to the topics from not only the contents (word frequencies used for each topic, e.g., the topic of democracy were more likely to use certain words in one period than than the other) but also the prevalence (topic frequencies used within a corpus, e.g., the topic of democracy was more mentioned in the thought emancipation period than in a certain period) and  of topics, which are two crucial perspectives of dynamics in concepts \citep{Lucas2015}.  Finally, because the study pays more attention to what the texts deliver rather than how they present it, the grammar and syntax, which the automated analysis is usually helpless with, no longer matters. Computer-assisted process can easily rule out these disturbing components from the corpus with a reliable preciseness. Also, it allows me to work on a large corpus of over a million articles, which will be considerably time-consuming by human-coding process.

Within the automated analytical methods, I choose the structural topic model (STM) to do most analyses in this study. The basic logic is formed in three steps as below, following the notation of \cite{Roberts2014a}. First, a logistic-normal general linear model ($\vec{\theta}_d$) is used to give a baseline distribution to the topic prevalent, which is allowed to vary based on the document covariates $X_d$ (the meta-data of the document, such as time, authorship, and language):
\begin{equation*}
\vec{\theta}_d | X_d\gamma, \Sigma \sim LogisticNormal(\mu =  X_d\gamma, \Sigma). 
\end{equation*} Then, for each word in each of the $d$ documents ($n \in [1, N_d]$), the distribution of the topic it assigned on conditioned by the distribution of the topic prevalent is assumed following a multinomial distribution: 
\begin{equation*}
z_{d,n} | \vec{\theta}_d \sim Multinomial(\vec{\theta}).
\end{equation*}
Finally, the observed words is also given a multinomial distribution conditioned by the topic chosen: 
\begin{equation*}
w_{d,n} | z_{d,n}, \beta_{d,k = z} \sim Multinomial(\beta_{d,k = z}), 
\end{equation*} where the topic distribution over words ($\beta_{d,k}$) is a combination of three components, a given baseline word distribution($ m $), the topic specific deviation ($\kappa_k$), the covariate group deviation ($\kappa_g$), and the topic-covariate interaction $\kappa_{i = kg_d}$ with a multinomial logit:
\begin{equation*}
\beta_{d,k} \varpropto exp(m + \kappa_k + \kappa_g + \kappa_{i = kg_d}). 
\end{equation*} Further details about the above process can be seen in \citep{Lucas2015,Roberts2014,Roberts2013}. 

The STM has all the qualities of the regular topic models (such as latent Dirichlet allocation (LDA) model and correlated topic model); one feature making STM superior to these models is that it allows incorporating meta-data associated with the documents into document-level covariates \citep{Lucas2015}. This feature is useful not only for studies which focus on the conditional effect of meta-data, but also for ruling out systematic differences within the corpus that are not of primary interest \citep[10]{Lucas2015}. In this study, it will be used to estimate the topics and, particularly, the correlations within them in the corpora. The test process is very straightforward: I calculate the correlations between each pair of topics. If the one correlates the democracy topic the most changes over time, \ref{h:dynamic} is supported; and if the most correlated topics correspond to the primary mission at that time period, \ref{h:response} is supported.

However, there are two things STM would not decide for me: how many topics I should estimate and how to identify the topics of democracy discourse and guardianship. Every topic model design has the first problem and there is no fixed ``correct'' answer to it. In this study, I take the suggestion given by \citep{Roberts2014} that the choice of topic number should rely on both the nature of the documents under study and the goals of the analysis. 

The main document source of this study is \textit{People's Daily}. This is the central committee official paper of CCP \citep{People}, and perhaps the most influential newspaper in China. It is required to be booked by every governmental, party departments, every level of schools, and even some enterprises and institutions, especially those state own ones. People, especially the officials and civil servants, do read it as a way to understand the attitudes of the authorities on general policies and specific issues. For Communist Party members, \textit{People's Day} also serves as an important source for learning and discussions in organizational lives\footnote{Organizational lives are the educational, organizational, and supervision activities to the party members in each party branch. It includes various formats, such as party day (weekly), party class (monthly), and party organizational life (seasonally or half a year). In these activities, the party members in a branch are usually organized to learn the recent important documents, instructions, or arguments sent by the Party Central Committee (often published in \textit{People's Daily}). Sometimes, members may also have a discussion about certain topics, often ones commented in the official publications, such as \textit{People's Daily}.} of party members. Finally, the reports and statements in \textit{People's Day} often become the basic or reference when other newspapers presenting the same issue or topic \citep{Roberts2015}. Considering its extensive and deep influences, this ought to be the primary place the Chinese political elites will go, if they intend to present or frame a concept in people's mind. Unfortunately, this important source has not been attract sufficient attention from political scientists of China study. So far, there are very few serious academic studies on \textit{People's Daily}, except for a couple of descriptive analyses from the communication and advertisement perspectives \citep{Chute1995,Swanson1996}. The most recent political scientific studies I can find was a ``research note'' published in last century \citep{Wu1994}.\footnote{At the same time, there are some political scientist studies on Chinese newspapers. However, most of them focus more on the provincial and city level semi-commercial newspapers \citep[for example, see ][]{Shirk2010,Reich2014,Roberts2015}.} One difficulty preventing scholars to make a systematical study on \textit{People's Daily} might be that the newspaper covers so many specific topics about Chinese political life and the focus and core meaning on certain topics may be different across time, although some terminologies did not change too much. This causes a tremendous problem to categorize  makes the texts of \textit{People's Daily} hard to be categorized based on limited terminology features. 

In this study, I solve this problem partially by STM. The method, as discussed above, does not require a clear pre-definition of the topic features. It will automatically assign words to its most popular topic(s) based on the probability they appeared in the corpus. Only the number of topic is required to be given. Considering the size of the \textit{People's Daily} corpus, it may be inappropriate and not very informational to estimate only three to five giant topics. On the other hand, the topics presented in newspapers, especially such standardized ones (i.e., the ones which formats and contents are both highly regularized and supervised) like \textit{People's Daily} should not as diverse as literatures or academic papers which require high creativities. Therefore, nor would I estimate over 50 topics as \citet{Blei2012} suggests on academic journal articles. Based on the above concerns, 15 to 20 can be a proper number of topics. This is also a relative acceptable number to do post-estimated analyses on each topics.

However, the topic number seems still a little small relative to a corpus including the over a million articles and across about 60 years. More importantly, single STM does not solve the problem to show the dynamic or changes in certain concepts across time. The way I deal with this issue is to do separate STMs in different time periods. By comparing relationships between topics in different period, I can show if the relationship is largely consistent or there is a dynamic. My corpus includes all the articles published in \textit{People's Daily} from 1946 to 2003. I divide them into five time periods, the ``pre-liberation'' period (the time before the founding of People's Republic of China, 1946-1949), the ``early-liberation'' period (1950-1965), and the ``Cultural Revolution'' period (1966-1977) of \textit{Mao's time}, the ``thought emancipation'' period (the period from the ``reform and open'' policy was promulgated till the June-$ 4^{th} $(Tian'anmen Square) event, 1978-1989) of \textit{Deng's time}, and the post-June-$ 4^{th} $ period of \textit{post-Deng time}. Each of the five periods of three times have distinctive characteristics and different governing missions being focused on. This can be seen from the words frequently used\footnote{In this study, I only analyze the content words (n. and adj.) because they are the most relative terms for construction of structural descriptive concepts like ``democracy.''} in the texts of these periods. \cref{f:cloudrmrb} show them by word clouds. From there, it is easy to see that, on the eve of liberation (\cref{f:cloud46}), the most frequent words used in \textit{People's Daily} is ``the People'' and ``Masses.'' No surprise that at that moment, the CCP intended to emphasize their political ideology and basis of governance favoring the majority of Chinese people, in order to gain their supports to build a new regime. However, this pattern partially changed when the regime entered maintaining step from the building step (\cref{f:cloud50}), although ``the People'' was still important, but ``working'' and ``production'' become the second most frequently used words, instead of ``Masses.'' In the next period, the changes became more obvious that ``Chairman'' (relating to the personal worship to Mao) and ``Revolution'' (Cultural Revolution) became the equally important words as ``the People.'' In the (pre-Tian'anmen) periods (when the reform and opening-up policy was started to implement), when the working focus of the government turned to economic development, the words ``economy,'' ``development,'' ``working,'', ``nation,'' and ``problem'' became the most frequently used words, and, finally, at the most recent periods, ``development'' became the dominant words used in the newspaper together with secondary frequently used words like ``enterprise,'' ``working,'' and ``building.''
\textbf{\begin{figure}[ht]
		\begin{subfigure}{0.33\textwidth}
			\includegraphics[width=\textwidth]{figure/cloud4649.png}
			\caption{Pre-Liberation}
			\label{f:cloud46}
		\end{subfigure}
		\begin{subfigure}{0.33\textwidth}
			\includegraphics[width=\textwidth]{figure/cloud5065.png}
			\caption{PRC Founding}
			\label{f:cloud50}
		\end{subfigure}
		\begin{subfigure}{0.33\textwidth}
			\includegraphics[width=\textwidth]{figure/cloud6677.png}
			\caption{Cultural Revolution}
			\label{f:cloud66}
		\end{subfigure}
		\begin{subfigure}{0.33\textwidth}
			\includegraphics[width=\textwidth]{figure/cloud7891.png}
			\caption{Pre-Tian'anmen}
			\label{f:cloud78}
		\end{subfigure}
		\begin{subfigure}{0.33\textwidth}
			\includegraphics[width=\textwidth]{figure/cloud9203.png}
			\caption{Post-Tian'anmen}
			\label{f:cloud92}
		\end{subfigure}
		\caption{Word Clouds of \textit{People's Daily} in Four Periods}
		\label{f:cloudrmrb}
	\end{figure}
	}
If the dynamic inferred from my hypotheses exists, there should be different characteristics in the concept of democracy between these periods. Also, if I estimate 15 topics in each period, I will estimate 75 topics in total, which is a reasonable number of topics in large corpus. After estimating the topics of each period, I calculate the correlation between them, controlling for the influences from the content in each topic, the prevalence across topics, the potential seasonal effect in months (e.g., there will be more reports about Spring Festival in February and more reports about agricultural harvests in fall months), and potential time dependence across years.

The only problem left is to identify the topics about democracy discourse and the guardianship characteristics. It is a core step of this empirical test, but, STM would not offer useful information on this. My strategy is to use the frequency of feature words to set consistent criteria. Particularly, the topic where ``democracy'' is the most frequently used is identified as the topic about the discourse of democracy. A similar strategy is applied to identify the topic about guardianship characteristics. As discussed in \cref{s:literature}, ``minben'' is the core quintessence of the guardianship essence of the regime. Hence, I use ``masses'' as the feature word for the guardianship-characteristic topic. For more general identification of the topic themes, I refer to both the most highly frequent words and simplified frequency exclusivity (FREX) scores \citep{Roberts2014a,Roberts2014,Lu2014a}, and make a subjective generalization. In the following section, I reports the results of the empirical studies.


\section{The Constructions of the Discourse of Democracy in Four Periods}
\subsection{Topic Models in \textit{People's Daily} Corpus}
Just from simple frequency counts, one can see that Chinese elites pay attention to the concept of democracy in all the periods but in different degrees. \cref{f:descDem} presents the frequencies of the word ``democracy'' and other two closely relevant words ``Democratization'' and ``democratic reform'' used across periods. ``Democracy'' was used at least over 8,000 times in every period (see \cref{f:descDemI} and also \cref{t:describ}). Nevertheless, the difference between the highest frequency to lowest frequency is over 26,000 times. The period the frequency was mostly used was the starting-up stage when the nondemocratic regime was established. This was also the period when the ``democratization'' and ``democratic reform'' was most frequently used (\cref{f:descDemII}). This trend suggests that the discourse of democracy was highly focused right after the founding of the regime. But did it only use to show the result of institutional contest, as Shi and their colleagues suggest? The STM results give a different answer.  

\begin{figure}[htbp]
	\begin{subfigure}{0.6\textwidth}
		\includegraphics[width=\textwidth]{figure/demodestriI.png}
		\caption{How Often ``Democracy'' Was Mentioned.}
		\label{f:descDemI}
	\end{subfigure}
	\begin{subfigure}{0.6\textwidth}
		\includegraphics[width=\textwidth]{figure/demodestriII.png}
		\caption{How Often ``Democra-'' Words Were Mentioned.}
		\label{f:descDemII}
	\end{subfigure}
	\caption{``Democracy'' and Relevant Concepts in \textit{People's Daily}}
	\label{f:descDem}
\end{figure}


\cref{t:toplist} lists the 15 topics identified in four periods' \textit{People's Daily}. Each periods includes one democratic topic and one guardianship topic. \cref{t:tmeg} gives an example of the correspondence between the frequent words and topic identification. After identifying the the topics, I calculate the correlations between each pair of topics in every period.\footnote{Because STM embed a Bayesian process, there is a risk that the posterior can be sensitive to the initialization. To minimize this risk, I run four STMs in each period with different random priors, then pick the one with the highest exclusivity and sematic coherence---in other words, the model with the lowest between-topic overlaps and highest within-topic consistency. The performances of the four models are visualized in \cref{f:mselermrb}. According to that, I chose model 4, 3, 4, 2, 4 respectively for each period.} \cref{f:corrmrb} visualizes these correlations. Each note represent a topic. Node size is proportional to the number of words in the corpus devoted to each topic. The edge width the lines connecting them represents how strong they are correlated to each other. Finally the color shows to what degree the topic is a seasonal topic---a topic that are more possible to appear in certain months than the other months of a year or multiple years.

\import{table/}{topic.list.tex}

\begin{figure}[htbp]
	\begin{subfigure}{0.45\textwidth}
		\includegraphics[width=\textwidth]{figure/cor4649.png}
		\caption{Pre-Liberation}
		\label{f:cor46}
	\end{subfigure}
	\begin{subfigure}{0.45\textwidth}
		\includegraphics[width=\textwidth]{figure/cor5065.png}
		\caption{PRC Founding}
		\label{f:cor50}
	\end{subfigure}
	\begin{subfigure}{0.45\textwidth}
		\includegraphics[width=\textwidth]{figure/cor6677.png}
		\caption{Cultural Revolution}
		\label{f:cor66}
	\end{subfigure}
	\begin{subfigure}{0.45\textwidth}
		\includegraphics[width=\textwidth]{figure/cor7891.png}
		\caption{Pre-Tian'anmen}
		\label{f:cor78}
	\end{subfigure}
	\begin{subfigure}{0.45\textwidth}
		\includegraphics[width=\textwidth]{figure/cor9203.png}
		\caption{Post-Tian'anmen}
		\label{f:cor92}
	\end{subfigure}
	\caption{Topic Correlations in \textit{People's Daily} in Four Periods}
	\label{f:corrmrb}
\end{figure}

These graphs give a vivid illustration of the dynamic in the corpus. Particularly, in \cref{f:cor92}, the topics separate to two groups. The correlations of topics within each group are very tight, while there is not cross-group connections at all. Comparing to it, topics in other four periods are all more or less mutually correlate to each other. 

Narrowing down to the dynamic in the discourse of democracy, first, the topics including the highest frequency of appearance of democracy are saliently different from period to period (see topics followed by ``(D)'' in \cref{t:toplist}). In the pre-liberation period when CCP needed extensive supports for the new regime, it was discussed in a topic about the decision making process within organizations (armies, families, or other groups). In the PRC founding period, democracy appeared most frequently in the topic about the strategies of the constructions of the new state. In the cultural revolution period when the CCP leaders were keen on domestic mass movement and international export of revolution, ``democracy'' was mostly mentioned in the topic of ``international communism.'' In the pre-Tian'anmen period when China decided to gradually introduce the market economy into the regime and open to the world, democracy was mostly discussed in the topic of collective ownership\footnote{Collective ownership is a ownership system between the private ownership and public ownership. The attention to it suggest CCP's endeavor to merge socialist public ownership and private ownership of market economy in that period.}. Finally, in the post-Tian'anmen period, democracy most frequently appeared in the topic about national development. In sum, the topics where democracy was frequently used highly corresponded to the primary work of the then period. 

\begin{figure}[htbp]
	\centering
	\caption{The Dynamics in the Discourse of ``Democracy''}\label{f:cordyn}
	\includegraphics[width=.8\textwidth]{figure/cordyn.png}
\end{figure}

Regarding the relationship to other topics, if the core function of the discourse of democracy is to show the result of institutional contest, it should enduringly highly correlate to the topic involving the guardianship characteristic. \cref{f:cordyn} shows a different pattern from this institutional contest theory. It presents the correlations of the democratic topic to the topic including guardianship characteristic and to the topics the democratic topic connecting the most tightly. If considering the correlation between a topic and the democratic topic as the contribution of the former to the latter, it is interesting that only in the pre-liberation period, the topic having the tightest correlation with the democratic topic is the one involving the guardianship characteristic. After then, the correlations between the two kept fluctuating between a range from 0.08 to 0.86. Again, my theory does not deny Shi and his colleague's finding that guardianship characteristic is an important contributor to the discourse of Chinese democracy. In the five periods covered by this study, the correlations between democratic topic and the guardianship-characteristic topic was over .60 in four periods. But at the same time, this is not the complete picture; there were always topics correlating to the democratic topic by over 0.85. In PRC founding period, it was about the topic about agricultural techniques corresponding to the primary governmental mission of recovering national economy from the war time and solving the subsistence (``food and clothing'') issue of the people. In Cultural Revolution, the highest correlated topic is agricultural development, corresponding to extreme left-thought in national development (for example, ``the (agricultural) yield increases as far the heart goes.''). In the pre-Tian'anmen period, democracy was highly correlated with the topic about redressment. The redressment was the a series of national decision to completely change the policy and decisions in the Cultural Revolution, which was the top one mission for CCP after the Cultural Revolution. Finally, in the post-Tian'anmen period, the highest correlated topic is the Chinese style development. After the ideological movement and the June $4^{th}$ crisis, CCP started to explore a developing path combining nondemocratic regime and market economy. The correlation to democratic topic corresponds to this primary mission. 

To sum up, the STM and post-estimation analyses show that the construction of the discourse of democracy in China involves much more diverse elements than just a declaration of the winning institutional setting. It undertakes the function to justify the primary missions of the authority through a political terminology. 

\subsection{Robustness Check}
A potential concern about the above analysis is that such corresponding relations with the discourse of democracy may appear only in the \textit{People's Daily} but not other corpus. In methodological terms, because the STM is neither an analysis on sample nor a parametric estimation, it can not offer an estimation of uncertainty. In this case, it is hard to tell to what degree the result is reliable. Regarding this concern, I do a robust check on another corpus, that is, the \textit{Selections} of the General Secretaries of the three generations of CCP leadership. The \textit{Selections of Mao Zedong} (\textit{Mao} in short hereafter) collects various articles of him published from 1925 to 1957; the \textit{Selections of Deng Xiaoping} (\textit{Deng} in short) collects his works from 1938 to 1992; and \textit{Selections of Jiang Zemin} (\textit{Jiang} in short) collects his work from 1980 to 2004. These \textit{Selections} include important articles, reports, and speeches of the heads of the three leaderships. This is an important data about political ideology and thoughts of modern China, which have not been systematically analyzed yet. In an authoritarian country like China, the leaders' expressions were not only about their own political opinions, but also set the main themes of policies and ideology of the then time. In this case, this source is almost equally important as \textit{People's Daily}, and therefor proper for robustness check. If the dynamic detected in \textit{People's Daily} is reliable, it should also appear in the corpus of the \textit{Selections}.

\begin{figure}[htbp]
	\begin{subfigure}{0.33\textwidth}
		\includegraphics[width=\textwidth]{figure/cloudmao.png}
		\caption{\textit{Selections of Mao}}
		\label{f:cloudmao}
	\end{subfigure}
	\begin{subfigure}{0.33\textwidth}
		\includegraphics[width=\textwidth]{figure/clouddeng.png}
		\caption{\textit{Selections of Deng}}
		\label{f:clouddeng}
	\end{subfigure}
	\begin{subfigure}{0.33\textwidth}
		\includegraphics[width=\textwidth]{figure/cloudjiang.png}
		\caption{\textit{Selections of Jiang}}
		\label{f:cloudjiang}
	\end{subfigure}
	\caption{Word Clouds of \textit{Selections}}
	\label{f:cloudsele}
\end{figure}

\cref{f:cloudsele} shows the frequent word clouds of the three \textit{Selections}. They suggest the themes of the works, and as the five periods of \textit{People's Daily}, the three \textit{Selections} also shows distinctive themes. The most frequently used words are ``the people'' and ``revolution'' for \textit{Mao}, ``problem'' and ``working'' for \textit{Deng}, and ``Development'' for \textit{Jiang}. \cref{t:describsele} shows some descriptive statistics about the discourse of democracy in these selections. Interestingly, \textit{Mao} used ``democracy'' and terms about democratic transform the most frequently in the three; and \textit{Deng} made by the author who started the political reform and market economy, discussed democracy and relevant the least. It may be because the June $4^{th}$ event happened when he was in charge and, since then, ``capitalist liberalization'' became an extreme negative term in Chinese politics. 


\import{table/}{description.sele.tex}

As doing on the \textit{People's Daily} corpus, I analyze the three \textit{Selections} separately in order to detect the dynamics through comparing the results. That means each corpus actually includes only one leader's selection. In this case, there is no need to use STM for prevalent factor controls. Instead, I use regular LDA modes with Gibbs sampling \citep{Griffiths2004,Phan2008}. Continuously applying the principle of choosing topic number accounting for both theory and the nature of the corpus, I chose 10 topics for all the three corpora.\footnote{This decision is coherent to data-driven approach of topic number choice based on  perplexity for the corpora of \textit{Deng} and \textit{Jiang} (see \cref{f:perplexity.sele}). The perplexity test for \textit{Mao} suggests 20 topics, but for the convenience to do post-estimate analysis and comparison, I still choose 10 topics for this corpus.} The ways to identify the topics and two special democratic-discourse and guardianship-characteristic topics is the same as did in \textit{People's Daily} corpus. 

\cref{t:topic.sele} shows the results. Democracy was most frequently mentioned in ``Goals of War'' in \textit{Mao}, ``Lesson Learnt [from Cultural Revolution]'' in \textit{Deng}, and ``China Style Economy'' in \textit{Jiang}. All of them largely primary missions in the then leadership generations. A potential problem of the \textit{Selection} corpus is that they were written by largely the same authorship \footnote{Although all the articles are undert the author's name of the core leader, they might not be written by only one person. Nevertheless, the written groups should be largely the same, and the articles were usually checked and proved by the leader before published.} in each selection. That suggests a high correlation among all or part of the articles. In this case, correlation analysis would not be an informational way. Instead, I use the network graphs to present their relations, as in \cref{f:lda.sele}.

\import{table/}{topic.sele.tex}

\begin{figure}[htbp]
	\begin{subfigure}{0.5\textwidth}
		\includegraphics[width=\textwidth, height = .3\textheight]{figure/ldamao.png}
		\caption{\textit{Selections of Mao}}
		\label{f:ldamao}
	\end{subfigure}
	\begin{subfigure}{0.5\textwidth}
		\includegraphics[width=\textwidth, height = .3\textheight]{figure/ldadeng.png}
		\caption{\textit{Selections of Deng}}
		\label{f:laddeng}
	\end{subfigure}
	\begin{subfigure}{0.5\textwidth}
		\includegraphics[width=\textwidth, height = .3\textheight]{figure/ldajiang.png}
		\caption{\textit{Selections of Jiang}}
		\label{f:ldajiang}
	\end{subfigure}
	\caption{Topic Correlations of \textit{Selections}}
	\label{f:lda.sele}
\end{figure}

As expected, because of the small size and potential homogeneity in the corpora, there are many shared words among topics. I then used the numbers of words between the first 10 frequent words in the topics to evaluate their correlations. In \textit{Mao}, there are two such words shared between the topic about democracy and the topic involving the guardianship characteristic. In \textit{Deng}, there are none. In \textit{Jiang}, there is only one. Instead, there are topics in both latter two corpora that sharing two words with the topic of democracy as those topics twined with the topic of democracy in \cref{f:lda.sele}. In a nut shell, what are found in the \textit{Selections} are largely consistent to the \textit{People's Daily} corpus. 


\section{Conclusion}

The empirical tests provide evidence to support the hypotheses implying the core function of the discourse of democracy is more to justify the contemporary primary missions of the authority rather than merely showing the institutional characteristics. There is no doubt that, since Mao Zedong to current Xi Jinping, Chinese elites have never stopped to differentiate ``China-style socialist democracy'' from ``western'' liberal democracy\citep{Lu2014a,Nathan1986}. Because this aspect always play a part in the construction of the discourse of democracy, it is easy to be captured, yet may be also easy to be overemphasized. As a pragmatist political party, CCP often use various means to serve for the contemporary primary missions. Following this logic, when a political crisis or their winning in the election would not been threatened by any factors making CCP to consider about the liberal democracy, it will automatically turn to that direction and deliver the liberal democracy value to the masses. 




\clearpage
\bibliographystyle{ajps}
\bibliography{E:/Dropbox_sync/Dropbox/Jabref}

\clearpage
\appendix
\appendixpage
\addappheadtotoc

\import{table/}{description.tex}
\import{table/}{topic.eg.tex}
\import{table/}{corDM.tex}






\begin{figure}[ht]
	\begin{subfigure}{0.45\textwidth}
		\includegraphics[width=\textwidth]{figure/msele4649.png}
		\caption{Pre-Liberation}
		\label{f:msele46}
	\end{subfigure}
	\begin{subfigure}{0.45\textwidth}
		\includegraphics[width=\textwidth]{figure/msele5065.png}
		\caption{PRC Founding}
		\label{f:msele50}
	\end{subfigure}
	\begin{subfigure}{0.45\textwidth}
		\includegraphics[width=\textwidth]{figure/msele6677.png}
		\caption{Cultural Revolution}
		\label{f:msele66}
	\end{subfigure}
	\begin{subfigure}{0.45\textwidth}
		\includegraphics[width=\textwidth]{figure/msele7891.png}
		\caption{Pre-Tian'anmen}
		\label{f:msele78}
	\end{subfigure}
	\begin{subfigure}{0.45\textwidth}
		\includegraphics[width=\textwidth]{figure/msele9203.png}
		\caption{Post-Tian'anmen}
		\label{f:msele92}
	\end{subfigure}
	\caption{STM Model Selections}
	\label{f:mselermrb}
\end{figure}


\begin{figure}[h!]
	\begin{subfigure}{0.33\textwidth}
		\includegraphics[width=\textwidth]{figure/perplexitymao.png}
		\caption{\textit{Selections of Mao}}
		\label{f:perplexitymao}
	\end{subfigure}
	\begin{subfigure}{0.33\textwidth}
		\includegraphics[width=\textwidth]{figure/perplexitydeng.png}
		\caption{\textit{Selections of Deng}}
		\label{f:perplexitydeng}
	\end{subfigure}
	\begin{subfigure}{0.33\textwidth}
		\includegraphics[width=\textwidth]{figure/perplexityjiang.png}
		\caption{\textit{Selections of Jiang}}
		\label{f:perplexityjiang}
	\end{subfigure}
	\caption{Topic Number Selections in the Topic Models of \textit{Selections}}
	\label{f:perplexity.sele}
\end{figure}

\end{document}
